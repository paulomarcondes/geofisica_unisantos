\documentclass[a4paper,11pt]{scrbook}


%\usepackage[latin1]{inputenc}
\usepackage[utf8x]{inputenc}
\usepackage[T1]{fontenc}
\usepackage[brazil]{babel}
\usepackage{indentfirst}
\usepackage{yfonts}
\usepackage[square]{natbib}
\usepackage{booktabs}
%\usepackage{pessoal}

%opening
\title{M\'etodos Geof\'isicos III (S\'ismica)}
\author{Paulo E. Pasquini Marcondes \\<pasquini.petrobras@gmail.com>}

\begin{document}

\maketitle
%\tableofcontents

%\begin{abstract}
%\end{abstract}

\chapter{Princ\'ipios e Limita\c{c}\~oes da Geof\'isica}

	\section{Introdu\c{c}\~ao}
A geof\'isica consiste na investiga\c{c}\~ao indireta do meio f\'isco (a Terra), com o fito entender ou explicar determinados fen\^omenos, desde a Tect\^onica de placas at\'e a forma\c{c}\~ao de depress\~oes nas ruas e avenidas de uma cidade.
Fazemos isso atrav\'es da medida de certos par\^ametros f\'isicos do meio, como sua densidade, resistividade, constante diel\'etrica, radioatividade, imped\^ancia el\'astica e muitos outros.
Cada m\'etodo geof\'isico \'e, portanto, controlado pelas leis f\'isicas do princ\'ipio no qual s baseia. Desta forma, n\~ao \'e poss\'ivel dominar, por exemplo, o M\'etodo Eletromagn\'etico sem dominar as Leis de Maxwell\footnote{James Clerk Maxwell, $\star$1831--1879$\dagger$}.

	\section{A Hist\'oria da Geof\'isica e seu contexto}
Atualmente, sempre que pensamos em geof\'isica, logo a associamos com a busca por min\'erios, petr\'oleo e com os terremotos.
A busca por min\'erios \'e muito antiga, recuando na noite dos tempos at\'e o Paleol\'itico.
Mas o registro cient\'ifico da minera\c{c}\~ao come\c{c}a com \emph{De Re Metallica} (Das Coisas Met\'alicas), publicado por Georgius Agricola, em 1556 \citep{telford_applied_1990}, que se tornou principal refer\^encia por muito anos.

O primeiro passo para a aplica\c{c}\~ao dos m\'etodos geof\'isicos provavelmente foi dado em 1843 quando Von Wrede percebeu que o teodolito magn\'etico de Lamont poderia ser usado na descoberta de corpos minerais magn\'eticos.
Entretanto, essa id\'eia n\~ao teve aplica\c{c}\~ao pr\'atica antes da publica\c{c}\~ao da obra \emph{On the Examination of Iron Ore Deposits by Magnetic Methods} publicada em 1879 por Robert Thal\'en.

Os avan\c{c}os na geof\'isica, assim como em todas as \'areas tecnol\'ogicas, est\'a profunda e inexoravelemente atrelado \'a ind\'ustria b\'elica.
Durante a Primeira Guerra Mundial, tentava-se localizar a posi\c{c}\~ao da artilharia inimiga atrav\'es das ondas de choque provocadas pelos disparos.
Estetosc\'opios era usados pelas equipes de sapadores para localizar as equipes inimigas, durante a fase conhecida como "Guerra de T\'uneis", como pode ser visto no filme "Pelot\~ao de Elite" \citep{sims_2010} (\emph{Beneath Hill 60}, no original em ingl\^es).
Na Segunda Guerra Mundial, a descoberta do RADAR, usando pulsos de r\'adio permitiu que os bombardeiros ingleses atingissem seus alvos mesmo em noites sem Lua e com tempo encoberto.
Posteriormente, o RADAR deu origem ao GPR (\emph{Ground Penetrating RADAR} -- RADAR de Penetra\c{c}\~ao no Solo), inicialmente usado na constru\c{c}\~ao de estradas no gelo no norte da Europa e posteriormente aplicado no estudo de \'areas contaminadas e nos estudos de an\'alogos para reservat\'orios de petr\'oleo.

Nos anos 1960, postulou-se que as anomalias magn\'eticas no assoalho oce\^anico, observadas desde os anos 1930, seriam a indica\c{c}\~ao de que os oceanos estavam crescendo \citep{vine_1965}. 
No ano seguinte, \citeauthor{wilson_1966} publica o artigo que marcaria o in\'icio da Teoria da Tect\^onica de Placas, cujo embri\~ao fora a Teoria da Deriva Continental, de \citet{van_der_gracht_theory_1928}.
Esta grande revolu\c{c}\~ao cient\'ifica foi motivada pela necessidade de um mapeamento detalhado do assoalho oce\^anico para que os rec\'em lan\c{c}ados submarinos nucleares pudessem permanecer escondidos nas profundezas.
Esse mapeamento estava sendo feito a partir das anomalias magn\'eticas.

De maneira semelhante, a dissemina\c{c}\~ao da s\'ismica multicanal e da migra\c{c}\~ao s\'ismica forneceram os subs\'idios que levaram a consolida\c{c}\~ao da Estratigrafia de Sequ\^encias, com o trabalho de \citeauthor{mitchum_seismic_1977} e outros naquele ano, imortalizados no famoso \emph{Memoir 26} da AAPG (\emph{American Association of Patroleum Geologists} -- Associa\c{c}\~ao Americana de Ge\'ologos de Petr\'oleo).
A Estratigrafia de Sequ\^encias estuda a evolu\c{c}\~ao dos sistemas e ambientes deposicionais ao longo to tempo e tem implica\c{c}\~ao para a defini\c{c}\~ao da paleogeografia e paleoclimatologia de uma \'area e ajuda a contar a hist\'oria evolutiva do Planeta, um dos focos da Geologia.

\section{Princ\'ipios F\'{i}sicos da Geof\'{i}sica} 
	Como o pr\'{o}prio nome implica, a geof\'{i}sica lida com a f\'{i}sica da Terra (e da atmosfera).
	Recentemente o nome tamb\'{e}m tem sido aplicado ao estudo indireto de outros corpos celestes, contexto no qual \'{e} mais comum vermos refer\^{e}ncia ao princ\'{i}pio f\'{i}sico ou fen\^{o}meno propriamente, como por exemplo na heliosismologia.
	Toda a geof\'{i}sica \'{e} baseada na exist\^{e}ncia de contrastes.
	Em outras palavras, a geof\'{i}sica \'{e} incapaz de detectar um objeto, mas apenas as diferen\c{c}as entre ele e os objetos ao redor.
	Nas palavras de \citeauthor{telford_applied_1990}:
	\begin{quote}
	It should be pointed out that geophysical techniques can detect only a discontinuity, that is, where on region differs sufficiently from another in some property.
	This, however, is a universal limitation, for we cannot perceive that which is homogeneous in nature; we can discern only that which is has some variation in time and/or space.
	\end{quote}
	
	Ent\~{a}o para que um objeto seja detect\'{a}vel, precisamos que haja contraste entre uma de suas propriedades e aquelas do meio circundante.
	Certas propriedades s\~{a}o percept\'{i}veis sem a necessidade de introduzirmos uma perturba\c{c}\~{a}o no meio.
	Outras, precisamos da perturba\c{c}\~{a}o para que sejam detect\'{a}veis.

	A primeira distin\c{c}\~{a}o que podemos fazer entre os m\'{e}todos geof\'{i}sicos, portanto, tem a ver com o tipo de fonte que vamos usar.
	Em outras palavras, posso perceber a propriedade desejada diretamente, ou preciso perturbar o meio?

	Os principais m\'{e}todos e suas respectivas fontes est\~{a}o sumariados nas Tabelas \ref{tab:fontesnat} e \ref{tab:fontesart}.

\begin{table}\centering
	\caption{M\'{e}todos Geof\'{i}sicos de fonte natural. Modificado de \cite{kearey_pt,telford_applied_1990}.}
	\begin{tabular}{lrr}
	\toprule
	\bfseries{Fonte}&\bfseries{M\'{e}todo}&\bfseries{Propriedade}\\
%	\cmidrule{c}{1-2}\\% \morecmidrules \cmidrule{c}{3-4}\\
	Campo Gravitacional&Gravimetria, gradiometria&Densidade\\
	Campo Magn\'{e}tico&Magnetometria&Susceptibilidade Magn\'{e}tica\\
	Calor & Geotermia & gradiente t\'{e}rmico\\
	Correntes Tel\'{u}ricas& Magnetotel\'{u}rico&?\\
	Radioatividade&M\'{e}todos Radioativos&radioatividade\\
	Tect\^{o}nica Global&Geod\'{e}sia e Gravimetria&deforma\c{c}\~{a}o da superf\'{i}cie\\
	\bottomrule
	\label{tab:fontesnat}
	\end{tabular}
\end{table}


\begin{table}\centering
	\caption{M\'{e}todos Geof\'{i}sicos de fonte artificial. Modificado de \cite{kearey_pt,telford_applied_1990}.}
	\begin{tabular}{lcr}
	\bfseries{Fonte}&\bfseries{M\'{e}todo}&\bfseries{Propriedade}\\
\toprule
%	\cmidrule{c}{1-2}\\% \morecmidrules \cmidrule{c}{3-4}\\
	Ondas Mec\^{a}nicas&S\'{i}smica&Imped\^{a}ncia El\'{a}stica\\
	Corrente El\'{e}trica& M\'{e}todos El\'{e}tricos&Resistividade El\'{e}trica\\
	\bottomrule
	\label{tab:fontesart}
	\end{tabular}
\end{table}

A escolha do m\'{e}todo geof\'{i}sico a empregar numa investiga\c{c}\~{a}o vai depender do conhecimento geol\'{o}gico pr\'{e}vio da \'{a}rea de estudo e do objeto de interesse.
Por isso, nos prim\'{o}rdios da ind\'{u}stria do petr\'{o}leo era comum a aplica\c{c}\~{a}o do m\'{e}todo gravim\'{e}trico.
Os exploracionistas de ent\~{a}o buscavam armadilhas formadas pelo dobramento das camadas perme\'{a}veis sobre os domos de sal
Desta forma, n\~{a}o buscavam diretamente as acumula\c{c}\~{o}es de hidrocarboneto, mas sim as estruturas geol\'{o}gicas que promoveriam a forma\c{c}\~{a}o do dep\'{o}sito.
Este tipo de racioc\'{i}nio \'{e} comum em toda a investiga\c{c}\~{a}o geof\'{i}sica.

\section{Limita\c{c}\~oes dos M\'etodos Geof\'isicos}

Os m\'etodos geof\'isicos s\~ao m\'etodos de investiga\c{c}\~ao indireta do meio f\'isico. 
Ou seja,

\bibliographystyle{agu}
\bibliography{../bib/unisantos,../bib/uni}
\end{document}
