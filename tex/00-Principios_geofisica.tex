\documentclass[a4paper,11pt]{scrbook}


% \usepackage[latin1]{inputenc}
%\usepackage[utf8x]{inputenc}
\usepackage[T1]{fontenc}
\usepackage[brazil]{babel}
\usepackage{indentfirst}
\usepackage{yfonts}
\usepackage[square]{natbib}
%\usepackage{pessoal}

%opening
\title{M\'etodos Geof\'isicos III (S\'ismica)}
\author{Paulo E. Pasquini Marcondes \\<pasquini.petrobras@gmail.com>}

\begin{document}

\maketitle
%\tableofcontents

%\begin{abstract}
%\end{abstract}

\chapter{Princ\'ipios e Limita\c{c}\~oes da Geof\'isica}

	\section{Introdu\c{c}\~ao}
A geof\'isica consiste na investiga\c{c}\~ao indireta do meio f\'isco (a Terra), com o fito entender ou explicar determinados fen\^omenos, desde a Tect\^onica de placas at\'e a forma\c{c}\~ao de depress\~oes nas ruas e avenidas de uma cidade.
Fazemos isso atrav\'es da medida de certos par\^ametros f\'isicos do meio, como sua densidade, resistividade, constante diel\'etrica, radioatividade, imped\^ancia el\'astica e muitos outros.
Cada m\'etodo geof\'isico \'e, portanto, controlado pelas leis f\'isicas do princ\'ipio no qual s baseia. Desta forma, n\~ao \'e poss\'ivel dominar, por exemplo, o M\'etodo Eletromagn\'etico sem dominar as Leis de Maxwell\footnote{James Clerk Maxwell, $\star$1831--1879$\dagger$}.

	\section{Hist\'oria da Geof\'isica}
Atualmente, sempre que pensamos em geof\'isica, logo a associamos com a busca por min\'erios, petr\'oleo e com os terremotos.
A busca por min\'erios \'e muito antiga, recuando na noite dos tempos at\'e o Paleol\'itico.
Mas o registro cient\'ifico da minera\c{c}\~ao come\c{c}a com \emph{De Re Metallica} (Das Coisas Met\'alicas), publicado por Georgius Agricola, em 1556 \citep{telford_1990}, que se tornou principal refer\^encia por muito anos.

O primeiro passo para a aplica\c{c}\~ao dos m\'etodos geof\'isicos provavelmente foi dado em 1843 quando Von Wrede percebeu que o teodolito magn\'etico de Lamont poderia ser usado na descoberta de corpos minerais magn\'eticos.
Entretanto, essa id\'eia n\~ao teve aplica\c{c}\~ao pr\'atica antes da publica\c{c}\~ao da obra \emph{On the Examination of Iron Ore Deposits by Magnetic Methods} publicada em 1879 por Robert Thal\'en.

Os avan\c{c}os na geof\'isica, assim como em todas as \'areas tecnol\'ogicas, est\'a profunda e inexoravelemente atrelado \'a ind\'ustria b\'elica.
Durante a Primeira Guerra Mundial, tentava-se localizar a posi\c{c}\~ao da artilharia inimiga atrav\'es das ondas de choque provocadas pelos disparos.
Estetosc\'opios era usados pelas equipes de sapadores para localizar as equipes inimigas, durante a fase conhecida como "Guerra de T\'uneis", como pode ser visto no filme "Pelot\~ao de Elite" \citep{sims_2010} (\emph{Beneath Hill 60}, no original em ingl\^es).
Na Segunda Guerra Mundial, a descoberta do RADAR, usando pulsos de r\'adio permitiu que os bombardeiros ingleses atingissem seus alvos mesmo em noites sem Lua e com tempo encoberto.
Posteriormente, o RADAR deu origem ao GPR (\emph{Ground Penetrating RADAR} -- RADAR de Penetra\c{c}\~ao no Solo), inicialmente usado na constru\c{c}\~ao de estradas no gelo no norte da Europa e posteriormente aplicado no estudo de \'areas contaminadas e nos estudos de an\'alogos para reservat\'orios de petr\'oleo.

Nos anos 1960, postulou-se que as anomalias magn\'eticas no assoalho oce\^anico, observadas desde os anos 1930, seriam a indica\c{c}\~ao de que os oceanos estavam crescendo \citep{vine_1965}. 
No ano seguinte, \citeauthor{wilson_1966} publica o artigo que marcaria o in\'icio da Teoria da Tect\^onica de Placas, cujo embri\~ao fora a Teoria da Deriva Continental, de Alfred Wegener, nos anos 1930.
Esta grande revolu\c{c}\~ao cient\'ifica foi motivada pela necessidade de um mapeamento detalhado do assoalho oce\^anico para que os rec\'em lan\c{c}ados submarinos nucleares pudessem permanecer escondidos nas profundezas.
Esse mapeamento estava sendo feito a partir das anomalias magn\'eticas.



\bibliographystyle{agu}
\bibliography{../bib/unisantos.bib}
\end{document}
