\documentclass[a4paper,11pt]{scrbook}


% \usepackage[latin1]{inputenc}
%\usepackage[utf8x]{inputenc}
\usepackage[T1]{fontenc}
\usepackage[brazil]{babel}
\usepackage{indentfirst}
\usepackage{natbib}
%\usepackage{pessoal}

%opening
\title{M\'etodos Geof\'isicos III (S\'ismica)}
\author{Paulo E. Pasquini Marcondes \\<pasquini.petrobras@gmail.com>}

\begin{document}

\maketitle
%\tableofcontents

%\begin{abstract}
%\end{abstract}

\chapter{Princ\'ipios da S\'ismica de Reflex\~ao}

Vamos abordar brevemente os princ\'ipios gerais do m\'etodo s\'ismico de reflex\~ao,
tal qual empregado na ind\'ustria do petr\'oleo.
As refer\^encias ao final do cap\'itulo fornecem informa\c{c}\~oes detalhadas e s\~ao leitura fortemente recomendada.
	
	\section{Introdu\c{c}\~ao}
O m\'etodo s\'ismico de reflexão envolve a excita\c{c}\~ao do meio f\'isico atrav\'es do uso de fontes s\'ismicas de v\'arias naturezas, o registro dos sinais recebidos, sua transformma\c{c}\~ao em imagens interpret\'aveis e, por fim a interpreta\c{c}\~ao propriamente dita.
Tendo em mente que os princ\'ipios aqui expostos tem o objetivo de fornecer uma vis\~ao geral do fen\^omeno, sigamos.

	\section{O Modelo Convolucional}
A convolu\c{c}\~ao \'e uma opera\c{c}\~ao matem\'atica, usada para representar o efeito de um filtro sobre um dado sinal.
Seja $s(t)$ um sinal cont\'inuo no tempo e $w(t)$ um filtro, o resultado da convolu\c{c}\~ao de ambos \'e:
\begin{equation}
S(t) = s(t) * w(t)
\end{equation}
Os geof\'isicos costumam dizer que a Terra funciona como um filtro \em{corta-altas}, ou seja, um filtro que deixa passar apenas as baixas frequ\^encias.
Desta forma, o sinal s\'ismico registrado pode ser entendido como a convolu\c{c}\~ao da assinatura da fonte com a resposta da Terra.
	\section{M\'etodo CMP}
A introdu\c{c}\~ao do m\'etodo CMP (\em{Common Mid Point} - Ponto M\'edio Comum), na d\'ecada de 1960, permitiu a gera\c{c}\~ao de imagens muito mais ricas, permitindo a descoberta de novos campos de petr\'oleo.
Inicialmente batizado por Wlliam Harry Mayne (\star 1913 \ndash 1990 \dagger), em 1962 de \em{Common Depth Point}, o grande avan\c{c}o do m\'etodo era permitir o empilhamento do sinal muito al\'em do ponto de satura\c{c}\~ao do equipamento de registro da \'epoca, anal\'ogico.
% Citar o paper Mayne(1962), da geophysics e o Dicionário do Oswaldão.
	\section{Imageamento em Tempo e em Profundidade}
\end{document}
