\documentclass[a4paper,11pt]{scrbook}


% \usepackage[latin1]{inputenc}
%\usepackage[utf8x]{inputenc}
\usepackage[T1]{fontenc}
\usepackage[brazil]{babel}
\usepackage{indentfirst}
\usepackage{yfonts}
\usepackage{natbib}
%\usepackage{pessoal}

%opening
\title{M\'etodos Geof\'isicos III (S\'ismica)}
\author{Paulo E. Pasquini Marcondes \\<pasquini.petrobras@gmail.com>}

\begin{document}

\maketitle
%\tableofcontents

%\begin{abstract}
%\end{abstract}

\chapter{Princ\'ipios e Limita\c{c}\~oes da Geof\'isica}

	\section{Introdu\c{c}\~ao}
A geof\'isica consiste na investiga\c{c}\~ao indireta do meio f\'isco (a Terra), com o fito entender ou explicar determinados fen\^omenos, desde a Tect\^onica de placas at\'e a forma\c{c}\~ao de depress\~oes nas ruas e avenidas de uma cidade.
Fazemos isso atrav\'es da medida de certos par\^ametros f\'isicos do meio, como sua densidade, resistividade, constante diel\'etrica, radioatividade, imped\^ancia el\'astica e muitos outros.
Cada m\'etodo geof\'isico \'e, portanto, controlado pelas leis f\'isicas do princ\'ipio no qual s baseia. Desta forma, n\~ao \'e poss\'ivel dominar, por exemplo, o M\'etodo Eletromagn\'etico sem dominar as Leis de Maxwell\footnote{James Clerk Maxwell, $\star$1831--1879$\dagger$}.

O m\'etodo s\'ismico de reflexão envolve a excita\c{c}\~ao do meio f\'isico atrav\'es do uso de fontes s\'ismicas de v\'arias naturezas, o registro dos sinais recebidos, sua transformma\c{c}\~ao em imagens interpret\'aveis e, por fim a interpreta\c{c}\~ao propriamente dita.

\end{document}
