\documentclass[a4paper,11pt]{scrartcl}


% \usepackage[latin1]{inputenc}
%\usepackage[utf8x]{inputenc}
\usepackage[T1]{fontenc}
\usepackage[brazil]{babel}
\usepackage{indentfirst}
\usepackage{natbib}
%\usepackage{pessoal}

%opening
\title{M\'etodos Geof\'isicos III (S\'ismica)}
\author{Paulo E. Pasquini Marcondes \\<pasquini.petrobras@gmail.com>}

\begin{document}

\maketitle
%\tableofcontents

% \begin{abstract}
% \end{abstract}

\begin{enumerate}
\item Princ\'ipios da S\'ismica de Reflex\~ao
	\begin{itemize}
	\item O Modelo Convolucional
	\item M\'etodo CMP 
	\item Imageamento em Tempo e em Profundidade
	\end{itemize}
\item Dispositivos de Aquisi\c{c}\~ao S\'ismica de Superf\'icie
	\begin{itemize}
	\item \emph{Split-spread, End-on};
	\item 2D;
	\item \emph{Streamer};
	\item 3D;
	\item NAZ, MAZ, WAZ;
	\item Helicoidal.
	\end{itemize}
\item Dispositivos de Aquisi\c{c}\~ao S\'ismica de Fundo
	\begin{itemize}
	\item Cabo de fundo;
	\item \emph{Nodes}.
	\end{itemize}
\item S\'ismica de Po\c{c}o
	\begin{itemize}
	\item \emph{Check-shot};
	\item Afastamento Zero (\emph{zero-offset});
	\item \emph{Walkaway};
	\item \emph{Walk-around;}
	\item 3D.
	\end{itemize}
\item Limita\c{c}\~oes do M\'etodo
	\begin{itemize}
	\item Resolu\c{c}\~ao Vertical
	\item Resolu\c{c}\~ao Horizontal
	\item \emph{Pullups e pulldowns}
	\item Acunhamentos
	\item Falhas
	\item Anisotropia
	\end{itemize}
\item Petrogeof\'isica
	\begin{itemize}
	\item Lei de Gassman
	\item Leis de Mistura
	\end{itemize}	
\item \emph{Amplitude versus Offset}
\item S\'ismica para Monitoramento de Reservat\'orios
	\begin{itemize}
	\item Modelagem
	\item Interpreta\c{c}\~ao
	\end{itemize}
\end{enumerate}
\end{document}
